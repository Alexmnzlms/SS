\chapter{Mi Segundo Modelo de Simulación de Monte Carlo}

\section{Modelización por montecarlo}

\subsection{Planteamiento del problema}
El sistema que vamos a modelizar trata de un quiosco de periódicos que se abastece diariamente de un cierto número de periódicos.

Por cada periódico vendido se obtiene una ganancia de x euros y por cada periódico no vendido se pierde una cantidad de y euros.

El número de periódicos solicitados al proveedor es s.

La demanda D es el número de periódicos que vende el quiosco cada día y obedece a una distribución de probabilidad P(D = d).

A continuación se definen las 3 distribuciones de probabilidad para D.

El objetivo de esta simulación es encontrar el número de periódicos que se solicitan --- el valor de s--- para el que la ganancia es máxima.

\subsection{Distribuciones de datos}

\begin{itemize}
	\item Distribución a: $P(D = d) = \frac{1}{100}$, $\forall 0,...,99$ \\ Esta es la distribución uniforme, es decir, todos los valores para D tiene la misma probabilidad.
	\item Distribución b: $P(D = d) = \frac{100 - d}{5050}$, $\forall 0,...,99$ \\ Esta es la distribución proporcional, los valores más pequeños de d tienen mayor probabilidad cuanto más pequeños sean.
	\item Distribución c: $P(D = d) = \frac{d}{2500}$ si $ 0 \leq d < 50$ y  $P(D = d) = \frac{100-d}{2500}$ si $ 50 \leq d \leq 99$ \\ Esta es la distribución triangular, lo que significa que los valores centrales de d tiene mayor probabilidad que los valores extremos, es decir, si $ 0 \leq d \leq 99$ entonces 0 y 99 tienen una probabilidad muy baja mientras que 50 --- que es el valor medio --- tiene la probabilidad más alta.
\end{itemize}



\subsection{Análisis de resultados}

A continuación pueden verse los resultados que se han obtenido para esta simulación variando le valor de y, es decir, modificando cuanto se pierde por cada unidad no vendida.
El número de iteraciones ha sido de 100000, lo que quiere decir, que para cada posible valor de s, se han generado 100000 valores aleatorios para D siguiendo las diferentes distribuciones de probabilidad.

\begin{table}[H]
\centering
\begin{tabular}{|c|c|c|c|c|}
\hline
\multicolumn{5}{|c|}{\textbf{Distribución a}} \\ \hline
\textbf{Veces} & \textbf{Ganancia/Unidad} & \textbf{Perdida/Unidad} & \textbf{Unidades} & \textbf{Ganancia} \\ \hline
100000 & 10 & 1 & 90 & 449.356 \\
100000 & 10 & 5 & 66 & 327.772 \\
100000 & 10 & 10 & 50 & 245.066 \\ \hline
\end{tabular}
\end{table}

\begin{table}[H]
\centering
\begin{tabular}{|c|c|c|c|c|}
\hline
\multicolumn{5}{|c|}{\textbf{Distribución b}} \\ \hline
\textbf{Veces} & \textbf{Ganancia/Unidad} & \textbf{Perdida/Unidad} & \textbf{Unidades} & \textbf{Ganancia} \\ \hline
100000 & 10 & 1 & 69 & 283.013 \\
100000 & 10 & 5 & 42 & 187.965 \\
100000 & 10 & 10 & 29 & 133.422 \\ \hline
\end{tabular}
\end{table}

\begin{table}[H]
\centering
\begin{tabular}{|c|c|c|c|c|}
\hline
\multicolumn{5}{|c|}{\textbf{Distribución c}} \\ \hline
\textbf{Veces} & \textbf{Ganancia/Unidad} & \textbf{Perdida/Unidad} & \textbf{Unidades} & \textbf{Ganancia} \\ \hline
100000 & 10 & 1 & 77 & 464.006 \\
100000 & 10 & 5 & 59 & 386.296 \\
100000 & 10 & 10 & 50 & 333.032 \\ \hline
\end{tabular}
\end{table}

Para la distribución a:

Podemos observar como, al aumentar el valor de pérdida/unidad, el valor óptimo de s disminuye, es decir, cuanto más aumenta lo que perdemos por una unidad, menos rentable resulta pedir muchas unidades, porque al ser la demanda igualmente probable, es igualmente posible vender 0 unidades que vender 99.

Para la distribución b:

En este caso podemos ver como, al ser más posible vender pocas unidades que vender mucho, es muy poco rentable encargar muchos periódicos cuanto mayor es la pérdida que obtenemos al dejar periódicos sin vender.
Por el mismo motivo, las ganancias máximás se reducen, puesto que lo más probable es vender pocos periódicos.

Para la distribución c:

Vemos que es bastante parecida a los resultados obtenidos para la distribución a. Esto tiene sentido puesto que la probabilidad de que salgan valores intermedios es mayor, por lo que de media obtendremos datos bastante parecidos.

\newpage

A continuación se pueden apreciar las gráficas de la ganancia obtenida para cada valor de s.


\begin{figure}[H]
	\centering
	\includegraphics[width=16cm]{distribucion_a_100000.png}
\end{figure}
\begin{figure}[H]
	\centering
	\includegraphics[width=16cm]{distribucion_b_100000.png}
\end{figure}
\begin{figure}[H]
	\centering
	\includegraphics[width=16cm]{distribucion_c_100000.png}
\end{figure}



\subsection{Tablas de datos}
A continuación se muestran los mismos datos, pero con variaciones en el número de ejecuciones que se realizan para cada valor de s.

\begin{table}[H]
\centering
\begin{tabular}{|c|c|c|c|c|}
\hline
\multicolumn{5}{|c|}{\textbf{Distribución a}} \\ \hline
\textbf{Veces} & \textbf{Ganancia/Unidad} & \textbf{Perdida/Unidad} & \textbf{Unidades} & \textbf{Ganancia} \\ \hline
100 & 10 & 1 & 89 & 463.86 \\
100 & 10 & 5 & 67 & 338.65 \\
100 & 10 & 10 & 53 & 251.8 \\
1000 & 10 & 1 & 90 & 453.356 \\
1000 & 10 & 5 & 68 & 329.135 \\
1000 & 10 & 10 & 51 & 241.68 \\
5000 & 10 & 1 & 91 & 454.153 \\
5000 & 10 & 5 & 68 & 330.752 \\
5000 & 10 & 10 & 49 & 239.772 \\
10000 & 10 & 1 & 90 & 451.672 \\
10000 & 10 & 5 & 67 & 330.668 \\
10000 & 10 & 10 & 50 & 246.502 \\
100000 & 10 & 1 & 90 & 449.356 \\
100000 & 10 & 5 & 66 & 327.772 \\
100000 & 10 & 10 & 50 & 245.066 \\ \hline
\end{tabular}
\end{table}

\begin{table}[H]
\centering
\begin{tabular}{|c|c|c|c|c|}
\hline
\multicolumn{5}{|c|}{\textbf{Distribución b}} \\ \hline
\textbf{Veces} & \textbf{Ganancia/Unidad} & \textbf{Perdida/Unidad} & \textbf{Unidades} & \textbf{Ganancia} \\ \hline
100 & 10 & 1 & 68 & 296.65 \\
100 & 10 & 5 & 43 & 196.3 \\
100 & 10 & 10 & 31 & 138.4 \\
1000 & 10 & 1 & 70 & 287.731 \\
1000 & 10 & 5 & 43 & 189.82 \\
1000 & 10 & 10 & 30 & 132.2 \\
5000 & 10 & 1 & 71 & 287.868 \\
5000 & 10 & 5 & 43 & 190.486 \\
5000 & 10 & 10 & 29 & 130.584 \\
10000 & 10 & 1 & 70 & 284.796 \\
10000 & 10 & 5 & 43 & 190.042 \\
10000 & 10 & 10 & 29 & 134.678 \\
100000 & 10 & 1 & 69 & 283.013 \\
100000 & 10 & 5 & 42 & 187.965 \\
100000 & 10 & 10 & 29 & 133.422 \\ \hline
\end{tabular}
\end{table}

\begin{table}[H]
\centering
\begin{tabular}{|c|c|c|c|c|}
\hline
\multicolumn{5}{|c|}{\textbf{Distribución c}} \\ \hline
\textbf{Veces} & \textbf{Ganancia/Unidad} & \textbf{Perdida/Unidad} & \textbf{Unidades} & \textbf{Ganancia} \\ \hline
100 & 10 & 1 & 77 & 472.23 \\
100 & 10 & 5 & 60 & 389.85 \\
100 & 10 & 10 & 52 & 334.4 \\
1000 & 10 & 1 & 79 & 466.281 \\
1000 & 10 & 5 & 60 & 385.515 \\
1000 & 10 & 10 & 51 & 330.16 \\
5000 & 10 & 1 & 79 & 467.379 \\
5000 & 10 & 5 & 59 & 383.501 \\
5000 & 10 & 10 & 50 & 329.704 \\
10000 & 10 & 1 & 79 & 465.595 \\
10000 & 10 & 5 & 60 & 387.565 \\
10000 & 10 & 10 & 50 & 334.46 \\
100000 & 10 & 1 & 77 & 464.006 \\
100000 & 10 & 5 & 59 & 386.296 \\
100000 & 10 & 10 & 50 & 333.032 \\ \hline
\end{tabular}
\end{table}

\newpage

\subsection{Gráficas}

Aquí podemos ver los datos de manera gráfica.

\begin{figure}[H]
	\centering
	\includegraphics[width=16cm]{distribucion_a_100.png}
	\includegraphics[width=16cm]{distribucion_a_1000.png}
\end{figure}
\begin{figure}[H]
	\centering
	\includegraphics[width=16cm]{distribucion_a_5000.png}
	\includegraphics[width=16cm]{distribucion_a_10000.png}
\end{figure}

\begin{figure}[H]
	\centering
	\includegraphics[width=16cm]{distribucion_b_100.png}
	\includegraphics[width=16cm]{distribucion_b_1000.png}
\end{figure}

\begin{figure}[H]
	\centering
	\includegraphics[width=16cm]{distribucion_b_5000.png}
	\includegraphics[width=16cm]{distribucion_b_10000.png}
\end{figure}

\begin{figure}[H]
	\centering
	\includegraphics[width=16cm]{distribucion_c_100.png}
	\includegraphics[width=16cm]{distribucion_c_1000.png}
\end{figure}

\begin{figure}[H]
	\centering
	\includegraphics[width=16cm]{distribucion_c_5000.png}
	\includegraphics[width=16cm]{distribucion_c_10000.png}
\end{figure}



\section{Modificaciones del modelo}

\subsection{Primera modificación}
La primera modificación al sistema consiste en eliminar las pérdidas por unidad no vendida, es decir, y = 0. Sin embargo introducimos un nuevo parámetro z que representa los gastos de devolución de las unidades no vendidas. Es decir, ahora las pérdidas son independientes de cuanto no vendamos, si no que dependen de cuanto cueste devolver las unidades no vendidas.

Podemos ver a continuación los resultados obtenidos en la simulación para esta modificación. Lo primero debe ser notar que los posibles valores de z con los que trabajamos son 100, 500 y 1000. Esto es respectivamente que los costes de devolución equivalen a 10 ventas --- porque por cada venta se obtiene un beneficio de 10 ---, 50 ventas y 100 ventas, es decir, un 10\% de las ventas, la mitad de las ventas y el total de las ventas, ya que la demanda máxima puede ser 100 periódicos.


\begin{table}[H]
\centering
\begin{tabular}{|c|c|c|c|}
\hline
\multicolumn{4}{|c|}{\textbf{Distribución a}} \\ \hline
\textbf{Ganancia/Unidad} & \textbf{Gastos Devolución} & \textbf{Unidades} & \textbf{Ganancia} \\ \hline
10 & 100 & 90 & 399.802 \\
10 & 500 & 49 & 127.062 \\
10 & 1000 & 0 & -0.302 \\ \hline
\end{tabular}
\end{table}

\begin{table}[H]
\centering
\begin{tabular}{|c|c|c|c|}
\hline
\multicolumn{4}{|c|}{\textbf{Distribución b}} \\ \hline
\textbf{Ganancia/Unidad} & \textbf{Gastos Devolución} & \textbf{Unidades} & \textbf{Ganancia} \\ \hline
10 & 100 & 80 & 235.934 \\
10 & 500 & 0 & -0.251 \\
10 & 1000 & 0 & -10.301 \\ \hline
\end{tabular}
\end{table}

\begin{table}[H]
\centering
\begin{tabular}{|c|c|c|c|}
\hline
\multicolumn{4}{|c|}{\textbf{Distribución c}} \\ \hline
\textbf{Ganancia/Unidad} & \textbf{Gastos Devolución} & \textbf{Unidades} & \textbf{Ganancia} \\ \hline
10 & 100 & 79 & 405.927 \\
10 & 500 & 36 & 204.343 \\
10 & 1000 & 24 & 119.323 \\ \hline
\end{tabular}
\end{table}

Podemos ver que para unos gastos de devolución de 100€ la ganancia aumenta según aumenta el número de periódicos que contratamos, porque a mayor número de periódicos tengamos, más probable es que vendamos mucho ---sobre todo para la distribución a y c --- y mayor beneficio obtendremos.

Para los gastos de devolución de 500€ podemos observar como el número de unidades de periódicos necesarias para alcanzar el mismo se reduce drasticamente, llegando a incluso a obtener pérdidas en vez de beneficios. Sobre todo para la distribución de probabilidad b, ya que es mucho menos probable vender un gran número de periódicos.

Para los gastos de devolución de 1000€ vemos que es una opción inviable para las distribuciones a y b, sin embargo, si es una opción posible si trabajamos con la distribución c, aunque los beneficios totales son bastante reducidos.

\newpage

A continuación podemos ver los resultados obtenidos gráficamente.

\begin{figure}[H]
	\centering
	\includegraphics[width=16cm]{mod_1_distribucion_a.png}
	\includegraphics[width=16cm]{mod_1_distribucion_b.png}
\end{figure}
\begin{figure}[H]
	\centering
	\includegraphics[width=16cm]{mod_1_distribucion_c.png}
\end{figure}

\subsection{Segunda modificación}

Finalmente la segunda modificación se basa en una combinación del problema original y la primera modificación.
Ahora es posible elegir entre la opción que genere la mínima pérdida de dinero, es decir, podemos escoger entre la opción más rentable entre pagar por cada unidad no vendida --- valor de y --- y pagar un precio fijo por la devolución del total de unidades no vendidas ---valor de z ---. En este caso los valores de z e y son los mismos que para los casos anteriores.

\begin{table}[H]
\centering
\begin{tabular}{|c|c|c|c|c|}
\hline
\multicolumn{5}{|c|}{\textbf{Distribución a}} \\ \hline
\textbf{Ganancia/Unidad} & \textbf{Perdida/Unidad} & \textbf{Gastos Devolución} & \textbf{Unidades} & \textbf{Ganancia} \\ \hline
10 & 1 & 100 & 90 & 448.202 \\
10 & 5 & 100 & 89 & 406.718 \\
10 & 10 & 100 & 89 & 401.75 \\
10 & 1 & 500 & 90 & 446.33 \\
10 & 5 & 500 & 66 & 325.459 \\
10 & 10 & 500 & 49 & 241.906 \\
10 & 1 & 1000 & 90 & 446.33 \\
10 & 5 & 1000 & 66 & 325.459 \\
10 & 10 & 1000 & 49 & 241.906 \\ \hline
\end{tabular}
\end{table}

\begin{table}[H]
\centering
\begin{tabular}{|c|c|c|c|c|}
\hline
\multicolumn{5}{|c|}{\textbf{Distribución b}} \\ \hline
\textbf{Ganancia/Unidad} & \textbf{Perdida/Unidad} & \textbf{Gastos Devolución} & \textbf{Unidades} & \textbf{Ganancia} \\ \hline
10 & 1 & 100 & 69 & 280.685 \\
10 & 5 & 100 & 71 & 234.661 \\
10 & 10 & 100 & 75 & 230.966 \\
10 & 1 & 500 & 69 & 280.685 \\
10 & 5 & 500 & 42 & 186.603 \\
10 & 10 & 500 & 29 & 131.972 \\
10 & 1 & 1000 & 69 & 280.685 \\
10 & 5 & 1000 & 42 & 186.603 \\
10 & 10 & 1000 & 29 & 131.972 \\ \hline
\end{tabular}
\end{table}

\begin{table}[H]
\centering
\begin{tabular}{|c|c|c|c|l|}
\hline
\multicolumn{5}{|c|}{\textbf{Distribución c}} \\ \hline
\textbf{Ganancia/Unidad} & \textbf{Perdida/Unidad} & \textbf{Gastos Devolución} & \textbf{Unidades} & \multicolumn{1}{c|}{\textbf{Ganancia}} \\ \hline
10 & 1 & 100 & 78 & 461.712 \\
10 & 5 & 100 & 72 & 412.534 \\
10 & 10 & 100 & 75 & 405.259 \\
10 & 1 & 500 & 78 & 461.712 \\
10 & 5 & 500 & 59 & 383.876 \\
10 & 10 & 500 & 50 & 329.892 \\
10 & 1 & 1000 & 79 & 461.563 \\
10 & 5 & 1000 & 59 & 382.647 \\
10 & 10 & 1000 & 50 & 329.892 \\ \hline
\end{tabular}
\end{table}

Podemos apreciar como en general, los resultados obtenidos se parecen significativamente a los obtenidos para la simulación sin modificar, por lo que podemos inferir, que resulta más económico fijar un precio de devolución variable segun el número de unidades que queden sin vender que fijar un precio total.

Esto tiene sentido, ya que al ser los precios de devolución por unidad más pequeños que los precios fijos de devolución, es decir, al ser y más pequeña que z, es muy improbable que la demanda sea lo suficientemente baja como para provocar unas pérdidas tan grandes por unidad no vendida que igualen o superen a las pérdidas producidas por un precio de devolución fijo.

Es también por este motivo que encontramos valores repetidos en las tablas y funciones superpuestas, ya que generalmente se aplica la función de ganancia utilizando el valor y en lugar del valor z.

\begin{figure}[H]
	\centering
	\includegraphics[width=16cm]{mod_2_distribucion_a.png}
	\includegraphics[width=16cm]{mod_2_distribucion_b.png}
\end{figure}
\begin{figure}[H]
	\centering
	\includegraphics[width=16cm]{mod_2_distribucion_c.png}
\end{figure}

\chapter{Generadores de datos}

\section{Mejorando los generadores}

En este apartado se pedía introducir una serie de mejoras de eficiencia en los generadores de datos aleatorios.

Dichas mejoras son las siguientes

\begin{itemize}
	\item Forma número 1: La forma número 1 es la de referencia, ya que es la manera en la que se han obtenido los datos en los apartados anteriores, es decir, recorriendo la tabla de probabilidad de forma secuencial.
	\item Forma número 2: Esta forma de mejora se basa en aplicar una búsqueda binaria para encontrar el valor aleatorio en la tabla de probabilidad.
	\item Forma número 3: Esta mejora es especifica para la distribución a y consigue obtener un tiempo constante de ejecución. Se basa en multiplicar por 100 el número obtenido entre 0 y 1 de manera uniforme. Dado que en la distribución a todos los valores poseen la misma probabilidad, simplemente multiplicando el valor uniforme aleatorio por el máximo, obtenemos el resultado requerido.
	\item Forma número 4: La forma número 4 de mejora es exclusiva para la distribución c y se basa en recorrer la tabla de probabilidad comenzando por los valores centrales y continuar en orden decreciente por la derecha y creciente por la izquierda. Así aseguramos comprobar primero los valores con mayor prioridad.
\end{itemize}

Podemos observar en las gráficas que se adjuntan a continuación:

Para la distribución a, vemos como la mejora 2 --- basada en búsqueda binaria --- consigue mejorar en eficiencia a la forma 1, obteniendo resultados casi un 50\% más rápido. Sin embargo vemos como la mejora 4 es imbatible, cosa que tiene sentido, ya que es capaz de obtener resultados en tiempo constante sin importar el tamaño de la tabla de búsqueda.

Para la distribución b observamos que la mejora 2, como es lógico, obtiene mejores resultados a la hora de encontrar los valores deseados.

Para la distribución c, podemos ver que la mejora 4, es significativamente más eficiente que la forma normal --- forma 1 ---, pero es superada por la mejora 2.

\begin{figure}[H]
	\centering
	\includegraphics[width=16cm]{tiempos_a.png}
	\includegraphics[width=16cm]{tiempos_b.png}
\end{figure}
\begin{figure}[H]
	\centering
	\includegraphics[width=16cm]{tiempos_c.png}
\end{figure}


\newpage
