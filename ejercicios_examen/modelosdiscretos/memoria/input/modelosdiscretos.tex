políticapolítica\chapter{Modelos discretos}

\section{Simulador base}
El código del simulador de esta compañía esta disponible en \textbf{src/compania.cpp} e \textbf{include/compania.h}

\subsection{Variables de interés}
\begin{itemize}
	\item \textbf{tam}: Representa el tamaño de la demanda del producto \textbf{D}.
	\item \textbf{nivel}: Representa el nivel de inventario \textbf{I(t)}.
	\item \textbf{pedido}: Representa el volumen de producto  que se va a pedir \textbf{Z}.
	\item \textbf{tultsuc}: Representa el tiempo ente el suceso actual y el suceso anterior.
	\item \textbf{spequena}: Representa \textbf{s}.
	\item \textbf{sgrande}: Representa \textbf{S}.
\end{itemize}

\subsection{Lista de sucesos}
La lista de sucesos se representa de la siguiente manera:
\begin{minted}{c++}
typedef struct {
	int suceso;
	float tiempo;
	registro reg_cola;
} suc;

suc nodo;

std::list<suc> lsuc;
\end{minted}
Simplemente se trata de una lista de objetos del tipo suc, que almacenan el tipo de suceso y el tiempo en el que está planificado.
Existe un suceso nodo, que se utiliza para almacenar el suceso actual en cada momento.

\newpage

\subsection{Grafo de sucesos}
\begin{figure}[H]
	\centering
	\includegraphics[width=0.8\textwidth]{flujo1.png}
	\caption{Grafo de sucesos}
\end{figure}

\subsection{Sucesos}
A continuación se muestran los sucesos que se han considerado:
\subsubsection{Suceso demanda}
Simula la llegada de una demanda del producto a la empresa.
\begin{minted}{c++}
if(nivel > 0){
	acummas += (reloj-tultsuc)*nivel;
} else {
	acummenos += (reloj-tultsuc)*(-nivel);
}
tultsuc = reloj;
tam = genera_tamano();
nivel -= tam;
nodo.suceso = SUCESO_DEMANDA;
nodo.tiempo = reloj+gendem(0.1);
insertar_lsuc(nodo);
\end{minted}
El nivel de inventario disminuye tanto como tamaño tenga la demanda del producto.

\subsubsection{Suceso evaluación de inventario}
Simula la revisión mensual del inventario por parte de la empresa.
\begin{minted}{c++}
if(nivel < spequena && pedido == 0){
	pedido = sgrande - nivel;
	acumpedido += K+i*pedido;
	nodo.suceso = SUCESO_PEDIDO;
	nodo.tiempo = reloj+genpedido(0.5, 1.0);
	insertar_lsuc(nodo);
}
nodo.suceso = SUCESO_EVAL;
nodo.tiempo = reloj+1.0;
insertar_lsuc(nodo);
\end{minted}

Si $I < s$ y no hay ninguna cantidad de pedido, entonces se hace un pedido de tamaño $S - I$.

\subsubsection{Suceso pedido}
Simula la llegada de un pedido encargado por la empresa.
\begin{minted}{c++}
if(nivel > 0){
	acummas += (reloj-tultsuc)*nivel;
} else {
	acummenos += (reloj-tultsuc)*(-nivel);
}
tultsuc = reloj;
nivel += pedido;
pedido = 0;
\end{minted}
El nivel de inventario aumenta según el tamaño del pedido.

\subsection{Generador de informes}
Las contadores estadísticos que se utilizan para la generación de informes son:
\begin{itemize}
	\item \textbf{acummas}: Contador estadístico que representa el coste de mantenimiento \textbf{ $I(t)^+$ }.
	\item \textbf{acummenos}: Contador estadístico que representa el coste de déficit \textbf{ $I(t)^-$ }.
	\item \textbf{acumpedido}: Contador estadístico que representa el coste de pedido \textbf{K+iZ}.
\end{itemize}

Los informes se almacenan en una matriz de datos, en los que cada fila es un linea del tipo:
\begin{minted}{c++}
	{(s,S), acumpedido + acummas + acummenos, acumpedido, acummas, acummenos};
\end{minted}

Además finalmente se muestra la combinación de (s,S) que obtiene un valor de coste total más pequeño.

Los valores aportados son la media de todas las simulaciones realizadas.

\subsection{Resultados de la simulación}
Simulaciones realizadas: 100000
% Please add the following required packages to your document preamble:
% \usepackage{graphicx}
\begin{table}[H]
\centering
\resizebox{\textwidth}{!}{%
\begin{tabular}{|c|c|c|c|c|}
\hline
\textbf{Política} & \textbf{Costo Total} & \textbf{Costo de pedido} & \textbf{Costo de mantenimiento} & \textbf{Costo de déficit} \\ \hline
(0,40) & 146.184 & 87.9343 & 5.80992 & 52.4401 \\
(0,60) & 135.859 & 84.3331 & 12.9841 & 38.5419 \\
(0,80) & 134.091 & 82.4005 & 21.2844 & 30.4064 \\
(0,100) & 136.499 & 81.2058 & 30.1149 & 25.1786 \\
(20,40) & 124.947 & 97.6004 & 9.12237 & 18.2241 \\
(20,60) & 118.934 & 88.5082 & 17.5176 & 12.9078 \\
(20,80) & 121.297 & 84.9072 & 26.8582 & 9.53105 \\
(20,100) & 126.9 & 82.9457 & 36.4555 & 7.4985 \\
(40,60) & 125.658 & 98.2922 & 25.5252 & 1.841 \\
(40,80) & 125.449 & 89.1253 & 34.9559 & 1.36733 \\
(40,100) & 131.483 & 85.4697 & 44.9903 & 1.02278 \\
(60,80) & 143.825 & 98.8523 & 44.9009 & 0.0722284 \\
(60,100) & 144.159 & 89.6499 & 54.455 & 0.0538317 \\ \hline
\end{tabular}%
}
\end{table}

El mínimo es 118.934 y se obtiene para la configuración (20,60).

\section{Primera modificación}
\subsection{Variables de interés}
\begin{itemize}
	\item \textbf{vendido\_mes}: Representa la cantidad de producto vendida en el mes anterior al actual.
\end{itemize}
\subsection{Grafo de sucesos}
\begin{figure}[H]
	\centering
	\includegraphics[width=0.8\textwidth]{flujo2.png}
	\caption{Grafo de sucesos}
\end{figure}

\subsection{Sucesos}
\subsubsection{Suceso demanda}

\begin{minted}{c++}
if(nivel > 0){
	acummas += (reloj-tultsuc)*nivel;
} else {
	acummenos += (reloj-tultsuc)*(-nivel);
}
tultsuc = reloj;
tam = genera_tamano();
nivel -= tam;
vendido_mes += tam;
nodo.suceso = SUCESO_DEMANDA;
nodo.tiempo = reloj+gendem(0.1);
insertar_lsuc(nodo);
\end{minted}
Se guarda la cantidad de producto vendido durante el mes actual.

\newpage

\subsubsection{Suceso evaluación de inventario}
\begin{minted}{c++}
pedido = vendido_mes;
vendido_mes = 0;
acumpedido += K+i*pedido;
nodo.suceso = SUCESO_PEDIDO;
nodo.tiempo = reloj+genpedido(0.5, 1.0);
insertar_lsuc(nodo);
nodo.suceso = SUCESO_EVAL;
nodo.tiempo = reloj+1.0;
insertar_lsuc(nodo);
\end{minted}
El pedido del mes es la cantidad de producto vendida el mes anterior.

\subsubsection{Suceso pedido}
No ha sido necesario realizar cambios en este suceso.

\subsection{Resultados de la simulación}
Simulaciones realizadas: 100000
% Please add the following required packages to your document preamble:
% \usepackage{graphicx}
\begin{table}[H]
\centering
\resizebox{\textwidth}{!}{%
\begin{tabular}{|c|c|c|c|c|}
\hline
\textbf{Política} & \textbf{Costo Total} & \textbf{Costo de pedido} & \textbf{Costo de mantenimiento} & \textbf{Costo de déficit} \\ \hline
0,40) & 135.501 & 105.974 & 29.0322 & 0.495184 \\
(0,60) & 135.51 & 105.988 & 29.027 & 0.494915 \\
(0,80) & 135.518 & 106.002 & 29.0191 & 0.496816 \\
(0,100) & 135.514 & 105.994 & 29.0242 & 0.495795 \\
(20,40) & 135.512 & 105.992 & 29.0239 & 0.496409 \\
(20,60) & 135.513 & 105.991 & 29.0245 & 0.496941 \\
(20,80) & 135.502 & 105.976 & 29.0317 & 0.494341 \\
(20,100) & 135.52 & 106.003 & 29.0191 & 0.497386 \\
(40,60) & 135.512 & 105.987 & 29.0295 & 0.494933 \\
(40,80) & 135.52 & 106.001 & 29.0199 & 0.499131 \\
(40,100) & 135.508 & 105.986 & 29.0272 & 0.495344 \\
(60,80) & 135.513 & 105.99 & 29.0259 & 0.496773 \\
(60,100) & 135.513 & 105.992 & 29.025 & 0.495176 \\ \hline
\end{tabular}%
}
\end{table}

El	mínimo es 135.501	y se alcanza en la configuración (0,40).

\newpage

\section{Segunda modificación}
\subsection{Variables de interés}
\begin{itemize}
	\item \textbf{pedido\_encargado}: Representa si hay encargado un pedido o no.
\end{itemize}

\subsection{Grafo de sucesos}
\begin{figure}[H]
	\centering
	\includegraphics[width=0.8\textwidth]{flujo3.png}
	\caption{Grafo de sucesos}
\end{figure}

\subsection{Sucesos}
\subsubsection{Suceso demanda}

\begin{minted}{c++}
if(nivel > 0){
	acummas += (reloj-tultsuc)*nivel;
} else {
	acummenos += (reloj-tultsuc)*(-nivel);
}
tultsuc = reloj;
tam = genera_tamano();
nivel -= tam;
if(nivel < spequena && !pedido_encargado){
	pedido = sgrande - nivel;
	acumpedido += K+i*pedido;
	nodo.suceso = SUCESO_PEDIDO;
	nodo.tiempo = reloj+genpedido(0.5, 1.0);
	insertar_lsuc(nodo);
	pedido_encargado = true;
}
nodo.suceso = SUCESO_DEMANDA;
nodo.tiempo = reloj+gendem(0.1);
insertar_lsuc(nodo);
\end{minted}
Siempre que no haya un pedido encargado e $ I < s $, se hace un pedido de tamaño $ S - I $.

\subsubsection{Suceso pedido}
\begin{minted}{c++}
if(nivel > 0){
	acummas += (reloj-tultsuc)*nivel;
} else {
	acummenos += (reloj-tultsuc)*(-nivel);
}
tultsuc = reloj;
nivel += pedido;
pedido = 0;
pedido_encargado = false
\end{minted}
Una vez llega el pedido, se puede volver a realizar otro.

\subsection{Resultados de la simulación}
Simulaciones realizadas: 100000
% Please add the following required packages to your document preamble:
% \usepackage{graphicx}
\begin{table}[H]
\centering
\resizebox{\textwidth}{!}{%
\begin{tabular}{|c|c|c|c|c|}
\hline
\textbf{Política} & \textbf{Costo Total} & \textbf{Costo de pedido} & \textbf{Costo de mantenimiento} & \textbf{Costo de déficit} \\ \hline
(0,40) & 125.901 & 92.8309 & 7.37193 & 25.6985 \\
(0,60) & 119.952 & 87.0459 & 15.5264 & 17.3799 \\
(0,80) & 121.883 & 84.2107 & 24.4896 & 13.1828 \\
(0,100) & 127.023 & 82.5862 & 33.8031 & 10.6338 \\
(20,40) & 124.722 & 106.049 & 11.5772 & 7.09565 \\
(20,60) & 117.737 & 93.4654 & 22.2841 & 1.9878 \\
(20,80) & 121.003 & 87.6403 & 32.0333 & 1.32936 \\
(20,100) & 127.627 & 84.7803 & 41.8365 & 1.00968 \\
(40,60) & 136.957 & 106.777 & 29.9181 & 0.262181 \\
(40,80) & 135.786 & 94.0802 & 41.6825 & 0.022782 \\
(40,100) & 139.859 & 88.2189 & 51.6256 & 0.0145748 \\
(60,80) & 157.185 & 107.478 & 49.7034 & 0.00345746 \\
(60,100) & 156.292 & 94.699 & 61.5931 & 2.88242e-05 \\ \hline
\end{tabular}%
}
\end{table}

El	mínimo es 117.737 y se alcanza en la configuración (20,60).

\newpage

\section{Conclusión}

Como se puede observar en los apartados correspondientes a los resultados de las simulaciones, la versión base y la segunda versión alcanzan el mínimo en la configuración (20,60) mientras que la primera versión lo hace en la (0,40).
Esto significa:
\begin{itemize}
	\item En la versión base, cada vez que se programa una evaluación, si el nivel del inventario es inferior a 20, se hace un pedido de 60 menos el nivel de inventario actual (se suma si existe déficit de existencias).
	\item La primera modificación depende de la demanda, no de la configuración.
	\item En la segunda modificación, implica que cada vez que el nivel del inventario es inferior a 20, sin necesidad de una evaluación y siempre que no haya otro pedido en curso, se hace un pedido de 60 menos el nivel de inventario actual (se suma si existe déficit de existencias).
\end{itemize}

Podemos descartar a la primera versión como una política que merezca la pena adoptar, puesto que es la que peores resultados aporta. No es conveniente dejarse llevar sólo por la demanda del mes anterior, sin realizar un análisis previo, a la hora de planificar las ventas del mes actual.

Por tanto, compararemos la versión base y la segunda versión. La diferencia entre estas versiones, no es más que el momento en el que se realiza un pedido. Para la versión base, cada vez se realiza una revisión del estado del inventario, se hace un pedido en consecuencia con el resultado de la evaluación. Sin embargo, la segunda versión no necesita esperar a realizar una evaluación y puede, de manera más dinámica, realizar un pedido siempre que el nivel de inventario rebaje un umbral de peligro (con la restricción de no poder hacer varios pedidos simultáneos).

Según los datos, la política más ventajosa es la que implementa la segunda versión, por lo que podemos inferir que una política  más dinámica es más favorable. Esto, si bien es cierto, es una afirmación un tanto pobre, ya que la mejora que obtiene la segunda versión respecto a la versión base es de 1.197€ de media en 100000 simulaciones. Esta diferencia es lo suficientemente pequeña como para no poder asegurar que en todos los casos, una política dinámica sea mejor que una política estática.
