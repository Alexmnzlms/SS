\chapter{Mi Tercer Modelo de Simulación Discreto}

\section{Modificaciones al sistema}
El código que contiene las modificaciones al sistema se encuentra disponible en \textbf{src/puerto.cpp}.

Para la modificación del remolcador al que no le afectan las tormentas, se ha optado por eliminar el primer suceso \textit{comienzo de tormenta} de manera que no se produzcan tormentas en el sistema.

\section{Comparativa de las modificaciones}
Tal y como se pide, se ha llamado una nueva medida de rendimiento llamada \textbf{Carga total}. Esta medida simplemente se calcula sumando el valor de carga correspondiente a cada tipo de carguero a una variable cada vez que sucede un evento \textit{fin de atraque}.

Los resultados obtenidos han sido obtenidos para 10000 simulaciones del modelo:

\begin{table}[H]
\centering
\begin{tabular}{|l|c|c|}
\hline
\multicolumn{1}{|c|}{\textbf{Modificación}} & \textbf{Media} & \textbf{Dt} \\ \hline
\textbf{Original} & 1.788.112,13 & 32.264,37 \\ \hline
\textbf{4 puntos de atraque} & 1.790.571,88 & 32.579,02 \\ \hline
\textbf{5 puntos de atraque} & 1.790.525,50 & 32.605,38 \\ \hline
\textbf{Remolcador al que no afectan las tormentas} & 1.788.556,88 & 32.181,05 \\ \hline
\textbf{Remolcador más rápido} & 1.787.903,63 & 31.801,75 \\ \hline
\end{tabular}
\end{table}

Podemos ver que según la carga total, la mejor modificación es la que consta de aumentar a 4 el número de puntos de atraque.
